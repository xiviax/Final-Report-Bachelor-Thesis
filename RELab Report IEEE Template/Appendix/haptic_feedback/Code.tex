\documentclass{article}
\usepackage{listings}
\usepackage{xcolor}
\lstset { %
    language=C++,
    backgroundcolor=\color{black!5}, % set backgroundcolor
    basicstyle=\footnotesize,% basic font setting
}

\begin{document}
This Code was used in order to read out the data via the Serial Monitor of Arduino IDE: \dots
\begin{lstlisting}
char *pressure_cells[] = { "Heel R:", "Heel L:", "Arch:", "Met 1:", 
"Met 3:", "Met 5:", "Hallux:", "Toes:"};

//static const uint8_t pins[] = {36,39,34,35,32,33,25,26}; 
//The Pins for the ESP 32
static const uint8_t pins[] = {0,1,2,3,4,5,6,7}; 
//The Pins for the Arduino Nano

#define LED_pin 6 //To test an LED-Feedback can be attached at Pin6 


#define Trigger_Point_Voltage_ON 0.2 
// This number has to be exceeded to trigger a feedback
#define Trigger_Point_Voltage_OFF 0.1 
// Measurement has to be below for a new feedback to be given


bool sense1_triggered=false;
float voltage_of_sensors[]={0.0, 0.0, 0.0, 0.0, 0.0, 0.0, 0.0, 0.0};

// the setup routine runs once when you press reset:

void setup() {
  // initialize serial communication at 115200 bits per second:
  Serial.begin(115200);

  pinMode(LED_pin, OUTPUT);
  
  Serial.println("Heel L;Heel R;Arch;Met 1;Met 3;Met 5;Hallux;Toes;"); 
  // for the excel sheet afterwards

}
// the loop routine runs over and over again forever:
void loop() {
  
  // Read out analog pins
  for (int i=0; i<8; i++)
  {
    // read the input on analog pin i:
    
    int sensorValue = analogRead(pins[i]);
    
    /* Convert the analog reading (which goes from 0 - 1023) to a 
    voltage (0 - 5V) for Arduino Nano */
    
    // Input should be from 0-1.4 V = 0 - 6000 mbar

    // float voltage = sensorValue * 5.0 / 1023.0 /  1.4 * 6000.0; 
    // To get the corresponding pressure
   
    voltage_of_sensors[i] = (sensorValue * 5.0 / 1023.0);

    Serial.print(voltage_of_sensors[i]);
    Serial.print(";");
    
    delay(10);
 
    /*if(voltage_of_sensors[i]>Trigger_Point_Voltage_ON && 
    sense1_triggered==false){
      digitalWrite(LED_pin, HIGH);
      delay(100);
      digitalWrite(LED_pin, LOW);
      sense1_triggered=true;
      }
    else if(voltage_of_sensors[i]<Trigger_Point_Voltage_OFF){
      
            sense1_triggered=false;
            
    }
    else{ //todo
      }*/ // If the LED wants to be tested.
  } 
 Serial.println("");

}
\end{lstlisting}

Crutch Code:

\begin{lstlisting}

\end{lstlisting}

Footsensor Code:

\begin{lstlisting}

\end{lstlisting}

Launch File:

\begin{lstlisting}

\end{lstlisting}

\end{document}